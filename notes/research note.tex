
\documentclass[12pt]{article}
%\pagenumbering{arabic} 

\usepackage[margin=1in]{geometry}
\usepackage{fancyhdr}
\pagestyle{fancy}

\usepackage{amsmath}
\usepackage{amssymb}

\newcommand{\diff}{\mathop{}\!d}
\DeclareMathOperator*{\argmin}{arg\,min}


\usepackage{kotex}
\usepackage{romannum}
\newtheorem{thm}{Theorem}
\newtheorem{cor}{Corollary}
\newtheorem{exa}{Example}
\newtheorem{ass}{Assumption}
\newtheorem{pro}{Proposition}
\newtheorem{defn}{Definitions}
\newtheorem{lem}{Lemma}
\newtheorem{pf}{proof}
\newtheorem{remark}{Remark}
\newtheorem{ex}{Exercise}
\newcommand*{\comb}[1][-1mu]{\permcomb[#1]{C}}
\lhead{Column Generation Techniques for GAP}
\rhead{김예린 연구노트}
%\rhead{}

\begin{document}
	\title{Column Generation Techniques for GAP}
	\date{\today}
	\maketitle
	
	\section{Literature Review}
	\begin{itemize}
		\item Branch-and-Price : Column Generation for Solving Huge Integer Programs (Barnhart, et. al., 1998)
		\item A Branch-and-Price Algorithm for the Generalized Assignment Problem (Savelsbergh and Martin, 1997) 
		\begin{itemize}
			\item Dantzig-Wolfe decomposition for GAP (construct master and sub problems)
			\item Column generation : additional columns of the restricted master problem are generated by solving the pricing problem.
			\item Branching strategies : variable dichotomy(sing variable) and GUB dichotomy(set of variables)
		\end{itemize}
		\item Chebyshev center based column generation (Lee and Park, 2011)
		\begin{itemize}
			\item The column generation procedure based on the simplex algorithm often shows desperately slow convergence. (zig-zag movement)
			\item Chebyshev center based column generation techniques
			\begin{itemize}
				\item Chebyshev center 
				\item Proximity adjusted Chebyshev center
				\item Chebyshev center + Stabilization
				\item Proximity adjusted Chebyshev center + Stabilization
			\end{itemize}
			\item Computational experiments on the binpacking, VRP, GAP
			\item The proposed algorithm could accelerate the column generation procedure.
		\end{itemize}
		\item Comparison of bundle and classical column generation (O.Briant, et. al., 2006 )
		\begin{itemize}
			\item Bundle method : the dual solution is often constrained to a given interval, and any deviation from the interval is penalized by a penalty function.
			\item The penalty function for stabilized column generation  : a simple V-shaped function (stabilizing center, $\epsilon$)
		\end{itemize}
		\item Stabilized Column Generation (O. Du Merle, et. al., 1997)
	\end{itemize}

	\section{Problems}
	\subsection{A case study : Generalized Assignment Problem }
	\paragraph{Dantzig-Wolfe Decomposition}\footnote{\scriptsize{The written mathematical formulation are from (Lee and Park, 2011)}}
	\begin{align*}
\text{(P)}~ \min& ~ \sum _ { i \in I } \sum _ { k \in K } c _ { k } ^ { i } x _ { k } ^ { i },\\
	\text { s.t. }&  ~ \sum _ { i \in I } \sum _ { k \in K _ { i } } \delta _ { k } ^ { j } x _ { k } ^ { i } \geq 1 , \quad j \in J, \\
	&- \sum _ { k \in K _ { i } } x _ { k } ^ { i } \geq - 1 , \quad \forall i \in I ,\\
	&x _ { k } ^ { i } \geq 0 , \quad \forall k \in K _ { i } , i \in I .	\\[5mm]
 	\text{(D)}~   \max& ~ \sum _ { j \in J } \pi _ { j } - \sum _ { i \in I } \phi _ { i }, \\
	\text{s.t. } & ~ \sum _ { j \in J } \delta _ { k } ^ { j } \pi _ { j } - \phi _ { i } \leq c _ { k } ^ { i } , \quad \forall k \in K _ { i } , i \in I, \\
	&\pi _ { j } \geq 0 , \quad \forall j \in J ,\\
	&\phi _ { i } \geq 0 , \quad \forall i \in I. 
	\end{align*}
	The GAP oracle finds an assignment pattern while satisfying the knapsack constraints :
	 \begin{equation*}
	 \max \sum _ { j \in J } \left( \pi _ { j } - c _ { i j } \right) \delta _ { j } , \quad \text { s.t. } \sum _ { j \in J } a _ { i j } \delta _ { j } \leq b _ { i } , \delta _ { j } \in \{ 0,1 \} , \quad \forall j \in J
	 \end{equation*}
	\paragraph{Stabilization}
	\begin{align*}
	(\tilde{P})~ \min& ~ \sum _ { i \in I } \sum _ { k \in K } c _ { k } ^ { i } x _ { k } ^ { i } + \sum_{ j \in J }\delta_j (\gamma_j^+ - \gamma_j^-) + \sum_{i \in I}\phi_i(y_i^+ - y_i^-),\\
	\text { s.t. }&  ~ \sum _ { i \in I } \sum _ { k \in K _ { i } } \delta _ { k } ^ { j } x _ { k } ^ { i } + \gamma_j^+ - \gamma_j^- \geq 1 , \quad \forall j \in J, \\
	&- \sum _ { k \in K _ { i } } x _ { k } ^ { i } +y_i^+ - y_i^-\geq - 1 , \quad \forall i \in I ,\\
	& \gamma_j^+ \leq \epsilon , ~ \gamma_j^- \leq \epsilon,\quad \forall j \in J,  \\
	& y_i^+ \leq \epsilon, ~ y_i^-  \leq \epsilon, \quad \forall i \in I, \\
	&x _ { k } ^ { i } \geq 0 , \quad \forall k \in K _ { i } , i \in I,\\
	&y_i^+\geq 0 , \quad \forall k \in K _ { i } , i \in I.
	\end{align*}
	
	\section{Preliminay Tests}
	Github page : \texttt{https://github.com/mody3062/CG}
	\paragraph{Testing algorithms}
	\begin{itemize}
		\item Classical column generation (Kelly's cutting plane)
		\item Stabilized column generation (O. Du Merle, et. al., 1997)
	\end{itemize}
	All codes for the both of algorithms are based on the pseudo code described in Figure 1 of (O. Du Merle, et. al., 1997). 
	
	
	\paragraph{Algorithmic parameters}
	
	RMP was constructed with a single decision variable which is dummy. The coefficient of the dummy variable on the objective function was set to a sufficiently large value, which is the sum of listed values such that \verb|np.sum(c,axis=1)|. For stabilized column generation algorithm, I fixed the parameter $\epsilon$ to 0.0001. ($\because$ I don't understand the criteria for changing the parameter value($\epsilon$). )
	
	
\end{document}